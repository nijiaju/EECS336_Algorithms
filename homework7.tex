\documentclass[12pt,letterpaper]{article}
\usepackage{amsmath, amssymb}
\usepackage{fullpage}
\usepackage{enumitem}
\usepackage{algorithmic}
\usepackage{algorithm}
\pagestyle{empty}
\def\pp{\par\noindent}

%%%%%%%%%%%%%%%%%%%%%%%%%%%%%%%%%%%%%%%%%%%%%%%%%%%%%%%%%%%%%%%%%%%%%%%%%%%%%%

\renewcommand{\baselinestretch}{1.2}
\newcommand{\problem}[1]{ \bigskip \pp \textbf{Problem #1}\par}
\newcommand{\solution}{\medskip\pp\textit{Solution:}\par}
\newcommand{\answer}{\medskip\pp\textit{Answer:} }
\newcommand{\lemma}[1]{\pp\textbf{\textit{Lemma #1:}}}
\newcommand{\proof}{\pp\textbf{\textit{Proof: }}}
\newcommand{\hint}[1] {\par{\footnotesize {\bf Hint:} #1}}
\newcommand{\remark}[1]{\par{\footnotesize {\bf Remark:} #1}}
\newcommand{\correctness}{\medskip\pp\textit{Correctness:}\par}
\newcommand{\timecomplexity}{\medskip\pp\textit{Time Complicity:}\par}
\renewcommand{\algorithmicrequire}{\textbf{Input: }}
\renewcommand{\algorithmicensure}{\textbf{Output: }}

%%%%%%%%%%%%%%%%%%%%%%%%%%%%%%%%%%%%%%%%%%%%%%%%%%%%%%%%%%%%%%%%%%%%%%%%%%%%%%

\newcommand{\bbZ}    {\mathbb{Z}}
\newcommand{\bbQ}    {\mathbb{Q}}
\newcommand{\bbN}    {\mathbb{N}}
\newcommand{\bbB}    {\mathbb{B}}
\newcommand{\bbR}    {\mathbb{R}}
\newcommand{\bbC}    {\mathbb{C}}
\newcommand{\calP}   {{\cal{P}}}

%%%%%%%%%%%%%%%%%%%%%%%%%%%%%%%%%%%%%%%%%%%%%%%%%%%%%%%%%%%%%%%%%%%%%%%%%%%%%%
\DeclareMathOperator{\E}{E}
\DeclareMathOperator{\Var}{Var}
\DeclareMathOperator{\cov}{cov}
%%%%%%%%%%%%%%%%%%%%%%%%%%%%%%%%%%%%%%%%%%%%%%%%%%%%%%%%%%%%%%%%%%%%%%%%%%%%%%

\begin{document}

\centerline{\bf EECS 336}

\medskip
\centerline{Jiaju Ni}
\centerline{Homework 7}
\bigskip


%%%%%%%%%%%%%%%%%%%%%%%%%%%%%%%%%%%%%%%%%%%%%%%%%%%%%%%%%%%%%%%%%%%%%%%%%%%%%%
% problem 1
\problem{1}
\begin{enumerate}
	\item
		\begin{enumerate}
			\item If $\alpha_{i-1}>1/2$ and $\alpha_i\geq1/2$, then
				\begin{align*}
					\widehat{c}_i&=c_i+\Phi_i-\Phi_{i-1}\\
					&=1+(2\cdot num_i-size_i)-(2\cdot num_{i-1}-size_{i-1})\\
					&=1+(2(num_{i-1}-1)-size_{i-1})-(2\cdot num_{i-1}-size_{i-1})\\
					&=-1
				\end{align*}
			\item If $\alpha_{i-1}=1/2$ and $\alpha_i<1/2$, then
				\begin{align*}
					\widehat{c}_i&=c_i+\Phi_i-\Phi_{i-1}\\
					&=1+(size_i/2-num_i)-(2\cdot num_{i-1}-size_{i-1})\\
					&=1+(size_{i-1}/2-(num_{i-1}-1))-(2\cdot num_{i-1}-size_{i-1})\\
					&=2+\frac{3}{2}size_{i-1}-3\cdot num_{i-1}\\
					&=2+\frac{3}{2}size_{i-1}-\frac{3}{2}size_{i-1}\\
					&=2
				\end{align*}
		\end{enumerate}
		Therefore, the amortized cost of \textsc{Table-Delete} is bounded above by a constant.
	\item
		The modified potential function is
		\[
			\Phi(T)=
			\begin{cases}
				8\cdot T.num-4\cdot T.size & \text{If } \alpha(T)\geq1/2\\
				5\cdot T.size-10\cdot T.num & \text{If } \alpha(T)<1/2
			\end{cases}
		\]
		\begin{enumerate}
			\item If $\alpha_{i-1}>1/2$ and $\alpha_i\geq1/2$, then
				\begin{align*}
					\widehat{c}_i&=c_i+\Phi_i-\Phi_{i-1}\\
					&=1+(8\cdot num_i-4\cdot size_i)-(8\cdot num_{i-1}-4\cdot size_{i-1})\\
					&=1+(8(num_{i-1}-1)-4\cdot size_{i-1})-(8\cdot num_{i-1}-4\cdot size_{i-1})\\
					&=-7
				\end{align*}
			\item If $\alpha_{i-1}=1/2$ and $\alpha_i<1/2$, then
				\begin{align*}
					\widehat{c}_i&=c_i+\Phi_i-\Phi_{i-1}\\
					&=1+(5\cdot size_i-10\cdot num_i)-(8\cdot num_{i-1}-4\cdot size_{i-1})\\
					&=1+(5\cdot size_{i-1}-10(num_{i-1}-1))-(8\cdot num_{i-1}-4\cdot size_{i-1})\\
					&=11+9\cdot size_{i-1}-18\cdot num_{i-1}\\
					&=11+9\cdot size_{i-1}-18\cdot\frac{1}{2}size_{i-1}\\
					&=11
				\end{align*}
		\end{enumerate}
		Therefore, the amortized cost of \textsc{Table-Delete} by the modified potential function is bounded above by a constant.
\end{enumerate}

% problem 2
\problem{2}
\begin{enumerate}
	\item
		\begin{enumerate}
			\item If $\alpha_{i-1}>1/2$ and $\alpha_i\geq1/2$, then
				\begin{align*}
					\widehat{c}_i&=c_i+\Phi_i-\Phi_{i-1}\\
					&=1+(2\cdot num_i-size_i)-(2\cdot num_{i-1}-size_{i-1})\\
					&=1+(2(num_{i-1}-1)-size_{i-1})-(2\cdot num_{i-1}-size_{i-1})\\
					&=-1
				\end{align*}
			\item If $\alpha_{i-1}=1/2$ and $\alpha_i<1/2$, then
				\begin{align*}
					\widehat{c}_i&=c_i+\Phi_i-\Phi_{i-1}\\
					&=1+(size_i-2\cdot num_i)-(2\cdot num_{i-1}-size_{i-1})\\
					&=1+(size_{i-1}-2(num_{i-1}-1))-(2\cdot num_{i-1}-size_{i-1})\\
					&=3+2\cdot size_{i-1}-4\cdot num_{i-1}\\
					&=3+2\cdot size_{i-1}-4\cdot\frac{1}{2}size_{i-1}\\
					&=3
				\end{align*}
			\item If $1/3<\alpha_{i-1}\leq1/2$ and $1/3\leq\alpha_i\leq1/2$, then
				\begin{align*}
					\widehat{c}_i&=c_i+\Phi_i-\Phi_{i-1}\\
					&=1+(size_i-2\cdot num_i)-(size_{i-1}-2\cdot num_{i-1})\\
					&=1+(size_{i-1}-2(num_{i-1}-1))-(size_{i-1}-2\cdot num_{i-1})\\
					&=3
				\end{align*}
			\item If $\alpha_{i-1}=1/3$, then
				\begin{align*}
					\widehat{c}_i&=c_i+\Phi_i-\Phi_{i-1}\\
					&=num_i+1+(size_i-2\cdot num_i)-(size_{i-1}-2\cdot num_{i-1})\\
					&=num_{i-1}-1+1+(\frac{2}{3}size_{i-1}-2\cdot(num_{i-1}-1))-(size_{i-1}-2\cdot num_{i-1})\\
					&=2-\frac{1}{3}size_{i-1}+num_{i-1}\\
					&=2-\frac{1}{3}size_{i-1}+\frac{1}{3}size_{i-1}\\
					&=2
				\end{align*}
		\end{enumerate}
		Therefore, the amortized cost of a \textsc{Table-Delete} that uses this strategy is bounded above by a constant.
	\item
		The potential function is $|1/2\cdot T.size-2\cdot T.num|$.
		\begin{enumerate}
			\item If $\alpha_{i-1}>1/4$ and $\alpha_i\geq1/4$, then
				\begin{align*}
					\widehat{c}_i&=c_i+\Phi_i-\Phi_{i-1}\\
					&=1+(2\cdot num_i-\frac{1}{2}size_i)-(2\cdot num_{i-1}-\frac{1}{2}size_{i-1})\\
					&=1+(2(num_{i-1}-1)-\frac{1}{2}size_{i-1})-(2\cdot num_{i-1}-\frac{1}{2}size_{i-1})\\
					&=-1
				\end{align*}
			\item If $\alpha_{i-1}=1/4$ and $\alpha_i<1/4$, then
				\begin{align*}
					\widehat{c}_i&=c_i+\Phi_i-\Phi_{i-1}\\
					&=1+(2\cdot num_i-\frac{1}{2}size_i)-(\frac{1}{2}size_{i-1}-2\cdot num_{i-1})\\
					&=1+(2(num_{i-1}-1)-\frac{1}{2}size_{i-1})-(\frac{1}{2}size_{i-1}-2\cdot num_{i-1})\\
					&=4\cdot num_{i-1}-size_{i-1}-1\\
					&=4\cdot\frac{1}{4}size_{i-1}-size_{i-1}-1\\
					&=-1
				\end{align*}
			\item If $1/6<\alpha_{i-1}\leq1/4$ and $1/6\leq\alpha\leq1/4$, then
				\begin{align*}
					\widehat{c}_i&=c_i+\Phi_i-\Phi_{i-1}\\
					&=1+(\frac{1}{2}size_i-2\cdot num_i)-(\frac{1}{2}size_{i-1}-2\cdot num_{i-1})\\
					&=1+(\frac{1}{2}size_{i-1}-2(num_{i-1}-1))-(\frac{1}{2}size_{i-1}-2\cdot num_{i-1})\\
					&=3
				\end{align*}
			\item If $\alpha_{i-1}=1/6$, then
				\begin{align*}
					\widehat{c}_i&=c_i+\Phi_i-\Phi_{i-1}\\
					&=num_i+1+(\frac{1}{2}size_i-2\cdot num_i)-(\frac{1}{2}size_{i-1}-2\cdot num_{i-1})\\
					&=num_{i-1}-1+1+(\frac{1}{2}\cdot\frac{2}{3}size_{i-1}-2(num_{i-1}-1))-(\frac{1}{2}size_{i-1}-2\cdot num_{i-1})\\
					&=num_{i-1}-\frac{1}{6}size_{i-1}+2\\
					&=\frac{1}{6}size_{i-1}-\frac{1}{6}size_{i-1}+2\\
					&=2
				\end{align*}
		\end{enumerate}
		Therefore, the amortized cost of a \textsc{Table-Delete} that uses this strategy is bounded above by a c    onstant.
\end{enumerate}

% problem 3
\problem{3}
Let $L_{opt}$ denotes an algorithm for the optimization problem \textsc{Longest-Path-Length} and $L_{dec}$ denotes an algorithm for the decision problem \textsc{Longest-Path}.\par
\lemma{1} If \textsc{Longest-Path} $\in P$, then the time complexity for $L_{opt}$ is $O(n^k)$ for constant $k$ where $n=|G|$.
\proof The algorithm $L_{opt}$ is as following (algorithm 1):
\begin{algorithm}
	\caption{$L_{opt}$}
\begin{algorithmic}[1]
	\REQUIRE $\langle G, u, v\rangle$
	\ENSURE the size of the longest path between vertex $u$ and vertex $v$ if it exists or $-1$ if it does not exist
	\STATE $k = |G| - 1$
	\WHILE {$k\geq0$ \AND $L_{dec}(G,u,v,k)=\FALSE$}
	\STATE $k=k-1$
	\ENDWHILE
	\RETURN $k$
\end{algorithmic}
\end{algorithm}\par
Since \textsc{Longest-Path} $\in P$, it is solvable in polynomial time. So, its time complexity is $O(n^{k'})$ for constant $k'$. The while loop has at most $n$ iterations and in each iteration we call $L_{dec}$ once. So, running time for $L_{opt}$ is $n^{k'+1}$ for constant $k'+1$.
\lemma{2} If \textsc{Longest-Path} $\notin P$, then $L_{opt}$ cannot be solved in polynomial time.
\proof The above algorithm shows that $T(L_{opt})=n\cdot T(L_{dec})$, so $L_{opt}$ cannot be sloved in polynomial time if \textsc{Longest-Path} $\notin P$.\par
From \textit{Lemma 1} and \textit{Lemma 2} we can conclude that the optimization problem \textsc{Longest-Path-Length} can be solved in polynomial time if and only if \textsc{Longest-Path} $\in P$.

% problem 4
\problem{4}
Let $f_{m\to l}$ denotes an algorithm converting a graph represented in adjacency matrix to which represented in adjacency list and $f_{l\to m}$ denotes the inverse algorithm of $f_{m\to l}$.\par
\begin{algorithm}
	\caption{$f_{m\to l}$}
	\begin{algorithmic}[1]
		\STATE $adj\_list \leftarrow$ an array of $|G|$ empty lists
		\FORALL {$v_1$ in $adj\_matrix$}
		\FORALL {$v_2$ in $adj\_matrix[v_1]$ such that $v_2\neq v_1$}
		\IF {$adj\_matrix[v_1][v_2]\neq\infty$}
		\STATE append $v_2$ to $adj\_list[v_1]$
		\ENDIF
		\ENDFOR
		\ENDFOR
	\end{algorithmic}
\end{algorithm}\par
\begin{algorithm}
	\caption{$f_{l\to m}$}
	\begin{algorithmic}[1]
		\STATE $n \leftarrow |G|$
		\STATE $adj\_matrix \leftarrow$ an $n\times n$ array with each element initialized to $\infty$
		\FORALL {$l_1$ in $adj\_list$}
		\FORALL {$v_2$ and $e$ in $l_1$}
		\STATE $adj\_matrix[l_1][v_2] \leftarrow e$
		\ENDFOR
		\ENDFOR
	\end{algorithmic}
\end{algorithm}\par
The above two algorithms (algorithm 2 and algorithm 3) can both run in $O(|G|^2)$ time. So there exist two polynomial-time computable functions $f_{m\to l}$ and $f_{l\to m}$ such that for any $g\in\text{ directed graph } G$, we have $f_{m\to l}(adj\_matrix(g))=adj\_list(g)$ and $f_{l\to m}(adj\_list(g))=adj\_matrix(g)$. Therefore, two representations are polynomially related.

% problem 5
\problem{5}
No. The dynamic-programming algorithm for the 0-1 knapsack problem runs in $\Theta(nW)$ time, where $n$ is the number of items and $W$ is the size of the knapsack. It is bound not only on the size of the input $n$ but also on the magnitude of the input $W$. If $W$ is encoded in a computer in binary, then
\begin{align*}
	T(n)&=\Theta(nW)\\
	&=\Theta(n2^{\log W})\\
	&=\Theta(n2^{\text{length of W}})
\end{align*}
If we add one bit in the length of code of $W$, then the running time will be doubled. So, it is not a polynomial-time problem.
\end{document}
