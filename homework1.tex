\documentclass[12pt,letterpaper]{article}
\usepackage{amsmath, amssymb}
\usepackage{fullpage}
\usepackage{amsmath}
\pagestyle{empty}
\def\pp{\par\noindent}

%%%%%%%%%%%%%%%%%%%%%%%%%%%%%%%%%%%%%%%%%%%%%%%%%%%%%%%%%%%%%%%%%%%%%%%%%%%%%%

\renewcommand{\baselinestretch}{1.2}
\newcommand{\problem}[1]{ \bigskip \pp \textbf{Problem #1}\par}
\newcommand{\solution}{\pp\textit{Solution:}\par}
\newcommand{\answer}{\medskip\pp\textit{Answer:} }
\newcommand{\lemma}[1]{\medskip\pp\textit{Lemma #1:}}
\newcommand{\proof}{\medskip\pp\textit{Proof:}\par}
\newcommand{\hint}[1] {\par{\footnotesize {\bf Hint:} #1}}
\newcommand{\remark}[1]{\par{\footnotesize {\bf Remark:} #1}}

%%%%%%%%%%%%%%%%%%%%%%%%%%%%%%%%%%%%%%%%%%%%%%%%%%%%%%%%%%%%%%%%%%%%%%%%%%%%%%

\newcommand{\bbZ}    {\mathbb{Z}}
\newcommand{\bbQ}    {\mathbb{Q}}
\newcommand{\bbN}    {\mathbb{N}}
\newcommand{\bbB}    {\mathbb{B}}
\newcommand{\bbR}    {\mathbb{R}}
\newcommand{\bbC}    {\mathbb{C}}
\newcommand{\calP}   {{\cal{P}}}

%%%%%%%%%%%%%%%%%%%%%%%%%%%%%%%%%%%%%%%%%%%%%%%%%%%%%%%%%%%%%%%%%%%%%%%%%%%%%%
\DeclareMathOperator{\E}{E}
\DeclareMathOperator{\Var}{Var}
\DeclareMathOperator{\cov}{cov}
%%%%%%%%%%%%%%%%%%%%%%%%%%%%%%%%%%%%%%%%%%%%%%%%%%%%%%%%%%%%%%%%%%%%%%%%%%%%%%

\begin{document}

\centerline{\bf EECS 336}

\medskip
\centerline{Jiaju Ni}
\centerline{Homework 1}
\bigskip


%%%%%%%%%%%%%%%%%%%%%%%%%%%%%%%%%%%%%%%%%%%%%%%%%%%%%%%%%%%%%%%%%%%%%%%%%%%%%%
% problem 1
\problem{1}
\begin{enumerate}
% 1
\item
\begin{solution}
$2^{2^{n+1}}=\Omega(n^22^{2n})$, $n^22^{2n}=\Omega((\frac{9}{4})^n)$, $(\frac{9}{4})^n=\Omega(n^{\lg\lg n})$, $n^{\lg\lg n}=\Omega(n^3)$, $n^3=\Omega(\lg n!)$, $\lg n!=\Omega(\lg^4n)$, $\lg^4n=\Omega(\ln^2n)$, $\ln^2{n})=\Omega(\ln^2\ln{n})$, $\ln^2\ln{n}=\Omega(\lg^*n)$, $\lg^*n=\Omega(1)$, $1=\Theta(n^{\frac{1}{\lg{n}}})$
\end{solution}
\begin{proof}
\begin{enumerate}
% 1.1
\item $2^{2^{n+1}}=\Omega(n^22^{2n})$\\
\begin{align*}
n^22^{2n} &\leq 2^n\cdot 2^{2n}\\
&=2^{3n} \text{ (for $n \geq 4$)}\\
&\leq 2^{2^{n+1}}
\end{align*}
Choose $c=1$, $n_0 = 4$, for any $n \geq n_0$ we have $n^22^{2n} \leq 2^{2^{n+1}}$,
$\therefore 2^{2^{n+1}}=\Omega(n^22^{2n})$.
% 2.1
\item $n^22^{2n}=\Omega((\frac{9}{4})^n)$
\begin{align*}
&\because \lim_{n\to\infty}\frac{n^22^{2n}}{\frac{9}{4}^n}
= \lim_{n\to\infty}n^2(\frac{16}{9})^n
= \infty\\
&\therefore n^22^{2n}=\Omega((\frac{9}{4})^n)\\
&\therefore n^22^{2n}=\Omega((\frac{9}{4})^n)
\end{align*}
% 3.1
\item $(\frac{9}{4})^n=\Omega(n^{\lg\lg n})$\\
% 4.1
\item $n^{\lg\lg n}=\Omega(n^3)$\\
Choose $c=1$, $n_0=10^{1000}$. For any $n>n_0$, we have $n^{\lg\lg n}>c\cdot n^3$, therefore $n^{\lg\lg n}=\Omega(n^3)$.
% 5.1
\item $n^3=\Omega(\lg n!)$\\
Because $n^3=\omega(n\ln n)$ (proof: $\lim_{n\to\infty}\frac{n^3}{n\ln n} = \lim_{n\to\infty} {2n^2} = \infty$), $n\ln n=\Theta(n\lg n)$, $n\lg n=\Theta(\lg n!)$, therefore $n^3=\Omega(\lg n!)$.
% 6.1
\item $\lg n!=\Omega(\lg^4n)$\\
Because $\lg n!=\Theta(n\lg n)$, $n\lg n=\omega(n)$, $n=\omega(\ln^4n)$ (proof: $\lim_{n\to\infty} \frac{n}{\ln^4n} = \lim_{n\to\infty}\frac{n}{4\ln^3n} = \lim_{n\to\infty}\frac{n}{12\ln^2n} = \lim_{n\to\infty}\frac{n}{24\ln n} = \lim_{n\to\infty}\frac{n}{24} = \infty$), $\ln^4n=\Theta(\lg^4n)$, therefore $\lg n!=\Omega(\lg^4n)$.
% 7.1
\item $\lg^4n=\Omega(\ln^2n)$\\
Because $\lg^4n=\omega(\lg^2n)$ (proof: $\lim_{n\to\infty} \frac{\lg^4n}{\lg^2n} = \lim_{n\to\infty} \lg^2n = \infty$), and $\lg^2n=\Theta(\ln^2n)$, therefore $\lg^4n=\Omega(\ln^2n)$.
% 8.1
\item $\ln^2{n}=\Omega(\ln^2\ln{n})$\\
Because $n=\omega(\ln{n})$ (proof: $\lim_{n\to\infty} \frac{n}{\ln n} = \lim_{n\to\infty} n = \infty$), we can get $\ln{n}=\omega(\ln\ln{n})$. Apply that again, we get $\ln^2{n}=\omega(\ln^2\ln{n})$, therefore $\ln^2{n}=\Omega(\ln^2\ln{n})$.
% 9.1
\item $\ln^2\ln{n}=\Omega(\lg^*n)$\\
Because $\ln^2\ln{n}=\omega(\ln\ln{n})$ (proof: $\lim_{n\to\infty}\frac{\ln^2\ln{n}}{\ln\ln{n}} = \lim_{n\to\infty} \ln\ln{n} = \infty$), $\ln\ln{n}=\Theta(\lg\lg{n})$ and $\lg\lg{n}=\Omega(\lg^*n)$ (proof by definition), we get $\ln^2\ln{n}=\Omega(\lg^*n)$.
% 10.1
\item $\lg^*n=\Omega(1)$\\
For any constant $c$, we can choose an $n_0$ which makes $\lg^*n > c$ for any $n > n_0$.
% 11.1
\item $1=\Theta(n^{\frac{1}{\lg{n}}})$
\begin{align*}
\because n^{\frac{1}{\lg{n}}} &= c\\
\frac{1}{\lg{n}}\lg{n} &= \lg{c}\\
c &= 10\\
\therefore 1 &= \Theta(n^{\frac{1}{\lg{n}}})
\end{align*}
\end{enumerate}
\end{proof}
\item
$|\tan(x)|$
\end{enumerate}

% problem 2
\problem{2}
\begin{enumerate}
% subproblem 1
\item $f(n)=\lg(\lg^*n)+2^{\lg n}\times(\lg n)^{\lg n}$
\begin{align*}
&\because 2^{\lg n}=\omega(\lg(\lg^*n)) \text{ and }(\lg n)^{\lg n}=\omega(\lg(\lg^*n))\\
&\therefore 2^{\lg n}\times(\lg n)^{\lg n}\text{ dominants.}\\
&f(n)=\Theta((2\lg n)^{\lg n})
\end{align*}
% subproblem 2
\item $f(n)=2^{\lg^*n}\times\lg((\lg n)^{\lg n})+(4^{\lg n})^3$
% subproblem 3
\item $f(n)=(\sqrt{2})^{\lg n}+e^n+\lg(\sqrt{\lg n})$\\
choose $c=1$ and $n_0=10$, for each $n>n_0$:
\begin{align*}
&e^n>(\sqrt{2})^{\lg n}>\lg(\sqrt{\lg n})>0\\
&\therefore e^n=\omega((\sqrt{2})^{\lg n}) \text{ and } e^n=\omega(\lg(\sqrt{\lg n}))\\
&\therefore e^n\text{ dominants.}\\
&\therefore f(n)=\Theta(e^n)
\end{align*}
% subproblem 4
\item $f(n)=n^2\times(4^{\lg n})^3+(\sqrt{2})^{\lg n}$
\begin{align*}
&\because (4^{\lg n})^3=2^{6\lg n}=\omega(2^{\frac{\lg n}{2}})=\omega((\sqrt{2})^{\lg n})\\
&\therefore n^2\times(4^{\lg n})^3\text{ dominants.}\\
&\therefore f(n)=\Theta(n^2\times(4^{\lg n})^3)=\Theta(n^8)
\end{align*}
% subproblem 5
\item $f(n)=n!\times(\lg n)!\times\sqrt{\lg n}+2^{\lg^*n}$
\begin{align*}
&\because n!=\omega(2^n),2^n=\omega(2^{\lg^*n})\\
&\therefore n!=\omega(2^{\lg^*n})\\
&\therefore n!\times(\lg n)!\times\sqrt{\lg n}\text{ dominants.}\\
&f(n)=\Theta(n!\times(\lg n)!\times\sqrt{\lg n})
\end{align*}
% subproblem 6
\item $f(n)=2^{(n+1)!}+(\sqrt{2})^{\lg n}$
\begin{align*}
&\because 2^{(n+1)!}=\omega(2^n), 2^n=\omega(2^{\lg n}),2^{\lg n}=\omega((\sqrt{2})^{\lg n})\\
&\therefore 2^{(n+1)!}=\omega((\sqrt{2})^{\lg n})\\
&\therefore 2^{(n+1)!}\text{ dominants.}\\
&f(n)=\Theta(2^{(n+1)!})
\end{align*}
\end{enumerate}

% problem 3
\problem{3}
\begin{enumerate}
% subproblem 1
\item
\begin{solution}
$f(n)=|n+tan(0.1\times n)|, g(n)=n$\\
\end{solution}
% subproblem 2
\item
\textbf{Advantage}\\
The "Omega infinity" extends the definition of "Omega", especially to some periodic function or non-continuous function. For example, in the answer of the previous subproblem, g(n) gives a meaningful asymptotic analysis of the lower bound of running time.\\
\textbf{Disadvantage}\\
If we have $f(n)=n$ and $g(n)=|n+tan(0.1\times n)|$, we still can find infinitely many integers $n$ such that $f(n)\geq cg(n)\geq 0$. However, this lower bound is not meaningful. In many integers $n$, $g(n)$ could be very large.
% subproblem 3
\item
In the direction of proof $\Omega$, nothing changes.\\
If we assume $f(n)>0$ (just like the text book did), then also nothing changes in the direction of proof "soft-oh".\\
If $f(n)$ is allowed to be negative, then its upper bound could be different from that of $O$. Its $O'$ upper bound would be $|g(n)|$.  ($f(n)=\Omega(g(n))$)
% subproblem 4
\item
\begin{align*}
\widetilde{\Omega}(g(n))=\{f(n): \text{ there exist positive constants $c$, $k$, and $n_0$ such that }\\
0\leq f(n)\leq cg(n)\lg^k(n)\text{ for all }n\geq n_0\}.\\
\widetilde{\Theta}(g(n))=\{f(n): \text{ there exist positive constants $c_1$, $k_1$, $c_2$, $k_2$ and $n_0$ such that}\\
0\leq c_1g(n)\lg^{k_1}(n)\leq f(n)\leq c_2g(n)\lg^{k_2}(n)\text{ for all }n\geq n_0\}.
\end{align*}
\lemma{1}
$f(n)=\widetilde{\Theta}(g(n))$ if and only if $f(n)=\widetilde{O}(g(n))$ and $f(n)=\widetilde{\Omega}(g(n))$
\begin{proof}
\begin{align*}
&\because f(n)=\widetilde{\Omega}(g(n))\text{ and }f(n)=\widetilde{O}(g(n))\\
&\therefore \text{ there exist positive constant $c_1$, $k_1$, $c_2$, $k_2$ and $n_1$, $n_2$ such that,}\\
&0\leq c_1g(n)\lg^{k_1}(n)\leq f(n)\text{ for all }n\geq n_1\text{ and }0\leq f(n)\leq c_2g(n)\lg^{k_2}(n)\text{ for all }n\geq n_2\\
&\text{choose } n_0=\max(n_1, n_2)\text{ we have: }\\
&0\leq c_1g(n)\lg^{k_1}(n)\leq f(n)\leq c_2g(n)\lg^{k_2}(n)\text{ for all }n\geq n_0\\
&\therefore f(n)=\widetilde{\Theta}(g(n))
\end{align*}
It is easy to proof in the other direction, just follow the definition, not repeat here.
\end{proof}
\end{enumerate}

% problem 4
\problem{4}
\begin{enumerate}
% a
\item $f(n)=n-1, c=2$\\
\begin{align*}
f_2^*(n)&=\min\{i\geq0: f^{(i)}(n)\leq2\}\\
&=\min\{i\geq0:n-i\leq2\}\\
&=n-2\\
&=\Theta(n)
\end{align*}
% c
\item $f(n)=n/2, c=3$
\begin{align*}
f_3^*(n)&=\min\{i\geq0: f^{(i)}(n)\leq3\}\\
&=\min\{i\geq0:\frac{n}{2^i}\leq3\}\\
&=\min\{i\geq0:n\leq3\cdot 2^i\}\\
&=\min\{i\geq0:i\geq\lg n-\lg 3\}\\
&=\lg n-\lg 3\\
&=\Theta(\lg n)
\end{align*}
% e
\item $f(n)=\sqrt{n}, c=4$
\begin{align*}
f_4^*(n)&=\min\{i\geq0: f^{(i)}(n)\leq4\}\\
&=\min\{i\geq0:n^{\frac{1}{2^i}}\leq4\}\\
&=\min\{i\geq0:\frac{1}{2^i}\lg n\leq2\}\\
&=\min\{i\geq0:2^i\geq\frac{1}{2}\lg n\}\\
&=\min\{i\geq0:i\geq\lg(\lg \sqrt{n})\}\\
&=\lg(\lg \sqrt{n})\\
&=\lg\lg n-1\\
&=\Theta(\lg\lg n)
\end{align*}
% g
\item $f(n)=n^\frac{1}{3}, c=4$
\begin{align*}
f_4^*(n)&=\min\{i\geq0: f^{(i)}(n)\leq4\}\\
&=\min\{i\geq0:n^{\frac{1}{3^i}}\leq4\}\\
&=\min\{i\geq0:\frac{1}{3^i}\lg n\leq2\}\\
&=\min\{i\geq0:3^i\geq\frac{1}{2}\lg n\}\\
&=\min\{i\geq0:i\geq\log_3(\lg \sqrt{n})\}\\
&=\log_3(\lg \sqrt{n})\\
&=\lg(\lg \sqrt{n})\\
&=\lg\lg n-1\\
&=\Theta(\lg\lg n)
\end{align*}
% h
\item $h(n)=\frac{n}{\lg n}, c=4$
\begin{align*}
&\because \lg n\geq2, \lg n\leq\sqrt{n}\text{ (for all $n\geq4$)}\\
&\therefore \frac{n}{\sqrt{n}}\leq\frac{n}{\lg n}\leq\frac{n}{2}\\
&\because f_4^*(\sqrt{n})=\Theta(\lg\lg n), f_4^*(\frac{n}{2})=\Theta(\lg n)\\
&\therefore f_4^*(\frac{n}{\lg n})=\Omega(\lg\lg n), f_4^*(\frac{n}{\lg n})=O(\lg n)
\end{align*}
\end{enumerate}
\end{document}
